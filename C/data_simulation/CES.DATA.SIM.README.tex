\documentclass{article}
\usepackage{fullpage}


\title{\textbf{CES--Codon Evolution Simulation} \\ {\Large Version 1.2}}

\author{Preston Hewgley, Micheal A. Gilchrist}

\begin{document}
\addtolength{\topmargin}{-1cm}
\addtolength{\textheight}{1cm}
\maketitle

%=================================================================================================================

\section{Overview}
	\begin{quote}
		CES (Codon Evolution Simulation) is a program that takes genetic sequences observed in the wild,
		 generates random synonymous\footnote{A synonym to a codon is a different codon that codes for the same 
		 amino acid. Since there are 61 translated codons and only 20 translated amino acids, there is
		 redundancy among codon-amino acid pairs. Thus, a synonymous sequence is a sequence that may 
		 have a different codon sequence, but has the same amino acid sequence.} sequences, and simulates 
		 the evolution of the random sequence under selective pressures against nonsense errors (NSE). 
		 CES simulates evolution by calculating the probability of mutant fixation\footnote{The probability of 
		 a mutant fixation is the probability that a mutation will occur and the new genotype will take over
		 the entire population.} for all one-step neighbors\footnote{One-step neighbors are synonymous
		 sequences that differ by only one codon.} and choosing a substitution based on these probabilities and a 
		 randomly generated number. By default, evolution steps are carried out until an average
		 of 5 substitutions per codon are carried out. The basis of the calculations used in this code are described in 
		 Sella and Hirsh (2005) and Gilchrist, Shah, and Zaretzki (2009). The code was used in Gilchrist,
		 Shah, and Zaretzki (2009) to generate simulated sequences and compare them to observed sequences
		 using several metrics. 

	\end{quote}

%=================================================================================================================	
	
\section{Repository Information}
	\begin{tabular}{|c|c|p{7cm}|}
		\hline
		\textbf{Description} & \textbf{Location} & \textbf{Notes} \\\hline
		Base Repository & file:///home/semppr/svnrepos/CES & The base repository for this project was made by Preston Hewgley in June 2012. \\\hline
		Source Files & file:///home/semppr/svnrepos/CES/source & This folder holds current and past versions. \\\hline
		Data & file:///home/semppr/svnrepos/CES/data & This folder holds FASTA files, elongation rates, expression levels, and mutation files. \\\hline
		Results & file:///home/semppr/svnrepos/CES/results & This folder holds data from results and graphics produces with the data.\\\hline
		Benchmarks & file:///home/semppr/svnrepos/CES/benchmark & This folder holds benchmark data produced using the -ben option.\\\hline
	\end{tabular}

%=================================================================================================================

\section{Commandline Arguments and Options}

	\subsection{Commandline Arguments}
	
		{\Large Here is a sample commandline template:}\\ \\
		{\renewcommand{\arraystretch}{1.5}
		 \renewcommand{\tabcolsep}{0.3cm}
		 \begin{tabular}{|c|}
			\hline
			./CES -F $<$\textit{FASTA input file}$>$ -T $<$\textit{tRNA/elongation rate file}$>$[\textit{Options}]\\
			\hline
		 \end{tabular}}
	
	\subsection{Options}
		{\renewcommand{\arraystretch}{1.3}
		 \renewcommand{\tabcolsep}{0.2cm}
		 \begin{tabular}{|c|c|p{10cm}|}
			\hline
			\textbf{Flag} & \textbf{Argument} & \textbf{Notes} \\
			\hline
			-F & $<$FILE$>$ & Location of the fasta file to be used for simulation. Note the program 
			will likely crash if genes with internal stop codons are used.
			phi values must be included on the same line as the ORF name OR included in a phi value file, e.g.\\
			& & 
			$>$YAL001C	phi=	0.0088862 \\
			& & Note the use of TABS between the ORF name and "phi=" and the phi value\\
			\hline
			-T & $<$STRING$>$ & Specify the location of the tRNA abundance/codon 
			translation file.\\
			%& & [DEFAULT] "tRNA\_files/S.cerevisiae.tRNA.tsv"\\
			\hline
			-P & $<$STRING$>$ & Specify the location of the phi value file. \\
			\hline
			-C & $<$NONE$>$ & Outputs codon counts for low expression genes. Does not produce output files for evolution simulation.\\
			\hline
			-U1 & $<$STRING$>$ & Specify the location of incoming sum mutation rate file (for more info see section \ref{subsec:this}).\\
			\hline
			-U2 & $<$STRING$>$ & Specify the location of individual mutation rate file (for more info see section \ref{subsec:this}).\\
			\hline
			-A1 & $<$REAL$>$ & Specify ribosome initation cost in ATPs \\
			& & [DEFAULT] a1=4\\
			\hline
			-A2  & $<$REAL$>$ & Specify ribosome elongation cost in ATPs,  a2 in SEMPPR \\
			& & [DEFAULT] a2=4\\
			\hline
			-AT & $<$REAL$>$ & Specify genome AT bias (0.0-1.0).\\
			& & [DEFAULT] at=0.5		\\
			\hline
			-B & $<$REAL$>$ & Specify the B parameter value\\
			& & [DEFAULT] B=0.0025\\
			\hline
			-D & $<$BOOLEAN$>$ & Indicate whether to print out delta\_eta files for each evolutionary step.
			Print outs can get quite large\\
			& & [DEFAULT] D=0.  \\
			\hline
			-E & $<$BOOLEAN$>$ & Indicate whether to print out eta and mu trace file\\
			& & [DEFAULT] E = 0. \\
			\hline
			-I & $<$INT$>$ & Specify whether CES should relax selection on the last $<$INT$>$ 
			amino acids of each sequence.  Used because it is thought that
			most genes can lose a few aa at the end, but still be 
			functional.\\
			\hline
			-M & $<$INT$>$ & Indicates simulation time. Simulation time is scaled by the mutation rate mu,  
			where mu = 1E-9. 
			If the argument X is $>$ 0, then we expect to see X * mu substitutions/codon.
			If X $<$ 0, then we run for -X/mu time or, in other words, we expect to see -X
			substitutions if the genes were evolving neutrally.\\
			& & [DEFAULT] = -20 \\
			\hline
			-Ne & $<$REAL$>$ & Effective population size. \\
			& & [DEFAULT] = 1.36E7\\
			\hline
			-O & $<$STRING$>$ & Specifies the folder for output files as well as prefix for specific
			output and log files.\\
			& & [DEFAULT] "output/out"\\
			\hline			
			-V & $<$REAL$>$ & Ratio of transition to transversion mutation rates.\\
			& & [DEFAULT] 1.0 \\
			\hline
			-ben & $<$NONE$>$ & Replaces randomly generated numbers with predetermined ones. 
			Useful when creating benchmarks or testing.\\
			\hline
		 \end{tabular}}
		
%=================================================================================================================	

\section{Input File Information}
	All acquired Input Files can be found in the data directory of CES.

	\subsection{FASTA file}
		\begin{itemize}
			\item File may start with header
			\item Each gene must start with ATG
			\item Must either (a) include phi values in header or (b) include phi value file
			\item Must be in proper FASTA file format, e.g. \\
				\begin{tabular}{c c c}
				$>$(Gene ID) & phi= & (Value) \\
				\multicolumn{3}{l}{ATGNNNNNNNNNNNN...}
				\end{tabular} 
		\end{itemize}
		
	\subsection{Phi value file}
		\begin{itemize}
			\item File may start with header, but the header must start with a double quote (")
			\item Gene ID and Phi value must be separated with "\verb+\t+", ",", or " " (tab, comma, or space).
			\item E.g.\\
				\begin{tabular}{c c}
				\multicolumn{2}{l}{"Optional Header"}\\
				YAL001C,0.008886154568855773
				\end{tabular}
		\end{itemize}
		
	\subsection{tRNA/Elongation file}
		\begin{itemize}
			\item File may start with header, but the header must start with a double quote(").
			\item Must separate AA index, Codon, and Elongation rate with "\verb+\t+" (tab).
			\item Must group synonymous codons together.
			\item E.g. \\
				\begin{tabular}{c c c}
				\multicolumn{3}{l}{"Optional Header"}\\
				A & GCA & 7.82\\
				\end{tabular}
		\end{itemize}
	
	\subsection{Mutation file}
		\label{subsec:this}
		There are two types of mutation files. The first type consists of the sum of incoming mutation rates for each codon.
		The second includes all individual mutation rates for all codons. 
		\subsubsection{Incoming Sum Mutation file}
			\begin{itemize}
				\item File may start with header, but the header must start with a double quote (").
				\item Must separate AA index, Codon, Mutation rate with "\verb+\t+" (tab).
				\item Must group synonymous codons together
				\item E.g. \\
					\begin{tabular}{c c c}
					\multicolumn{3}{l}{"Optional Header"} \\
					A & GCA & 1 \\
					A & GCC & 0.5872495 \\
					\end{tabular}
			\end{itemize}
		\subsubsection{Individual Rate Mutation file}
			\begin{itemize}
				\item File may start with header, but the header must start with a double quote(").
				\item Must conform to this format: \\
					\begin{tabular}{c c c c c}
						\multicolumn{5}{l}{"Optional Header"} \\
						$>$A & GCA & GCC & GCG & GCT \\
						GCA & 0 & .22 & .11 & .32 \\
						GCC & .34 & 0 & .13 & .30 \\
						GCG & .30 & .25 & 0 & .29 \\
						GCT & .32 & .27 & .09 & 0 \\
						$>$C & TGC & TCT \\
						TGC & 0 & 2.09 \\
						TGT & 1 & 0 \\
					\end{tabular} \\
					ETC...
				\item A "$>$" must preeced each Amino Acid.
				\item Entries may be separated with "\verb+\t+" or " " (tab or space).
				\item Index convention is as follows: Entry in row i and column j is mutation rate from 
					  codon i to codon j ($\mu_{ij}$).				
				\item Mutation rate of any codon to itself is 0.
				\item Rows and columns must be labeled symmetrically. Check if diagonal is composed of zero's.

			\end{itemize}
				

%=================================================================================================================	
	
\section{Pseudocode}

	\subsection{Quick and Dirty}
		\begin{enumerate}
			\item Read Input Files (Fasta sequence file, elogation rate file, and (optional) mutation rate file)
			\item Generate random synonymous sequence
			\item Calculate $\eta_{obs}$
			\item Calculate $\Delta\eta$ for all one-step neighbors
			\item Calculate Replacement Probability for all one-step neighbors
			\item Choose random number to determine substitution based on Replacement Probability
			\item Repeat steps 3-6 for desired evolution time
			\item Print output
		\end {enumerate}

%=================================================================================================================		
	
		
\section{Overview of Equations}
	\begin{center}	
	
	\begin{tabular}[c]{r |l|}
		\cline{2-2}
		{\Large $\sigma_{i}(\vec{c})  =  \prod_{j=1}^i \frac{c_{j}}{(c_{j} + b)} \qquad \qquad$} & Translation Probability \\ \cline{2-2}
	\end{tabular} \\\
	
	\begin{tabular}{c r @{=} l}
		Where: & $c_{j}$ & Elongation rate of codon j \\
		& $b$ & Background nonsense error rate \\
	\end{tabular}
		
	\end{center}
	
	\begin{quote}
		This is the probability that a ribosome will translate up to and including codon i. Since it can be shown that the probability of 
		translating a single codon j equals $\frac{c_{j}}{c_{j}+b}$, it follows that the cumulative translation probability of codon i equals 
		the product of the elongation probability for every codon up to and including codon i. \textbf{NOTE:} $\sigma_{n}$ denotes the 
		probability that a protein is fully translated. \\ 
	\end{quote} 
	
	\begin{center}
	
	\begin{tabular}[c]{c|l|}
		\cline{2-2}
		{\Large $\xi(\vec{c}) = \frac{1}{1-\sigma_{n}(\vec{c})}\sum_{i=1}^n(a_{1}+a_{2}i)\sigma_{i-1}(\vec{c})\frac{b}{c_{i}+b}  \qquad \qquad$} & Expected energetic\\
		&  cost of a NSE \\
		\cline{2-2}	
	\end{tabular}\\\
	
	\begin{tabular}{c r @{=} l}
		Where: & $a_{1}$ & Energetic cost of ribosome recharge\\
		& $a_{2}$ & Energetic cost of a peptide bond\\
	\end{tabular}\\*
	
	\end{center}
	
	\begin{quote}
		Note that the first part of the summation, $(a_{1}+a_{2}i)$, is the cost of translating up to codon i and the second part,
		$\sigma_{i-1}(\vec{c})\frac{b}{c_{i}+b}$, is the probability that a nonsense error will occur on codon i. $\frac{1}{1-\sigma_{n}(\vec{c})}$
		is a factor to scale this calculation. \\
	\end{quote} 
	
	\begin{center}
	
	\begin{tabular}[c]{r |l|}
		\cline{2-2}
		{\Large $\eta(\vec{c}) = \bigg(\frac{1}{\sigma_{n}}-1\bigg)\xi(\vec{c}) + (a_{1}+a_{2}i)  \qquad \qquad$} & Expected cost of\\
		& producing one protein\\
		\cline{2-2}
	\end{tabular}\\\
	
	\end{center}
	
	\begin{quote}
		Essentially, this is a composition of the expected cost of all nonsense errors plus the cost of a completed protein,
		$(a_{1}+a_{2}i)$. \\
	\end{quote} 
	
	\begin{center}
	
	\begin{tabular}[c]{r |l|}

		\cline{2-2}
		{\Large $w(\vec{c}) \propto e^{-q\phi\eta}  \qquad \qquad$} & Fitness function\\
		\cline{2-2}
	\end{tabular}\\\
	
	\begin{tabular}{c r @{=} l}
		Where: & $q$ & Scaling factor\\
		& $\phi$ & Protein production rate\\
	\end{tabular}\\
	
	\end{center}
	
	\begin{quote}
		We assume that fitness is proportional to a negative exponential function of the protein cost rate. \\
	\end{quote} 
	
	\begin{center}
	
	\begin{tabular}[c]{r |l|}
		\cline{2-2}
		{\Large $ \pi(i \rightarrow j) = \frac{1-{\huge\frac{w_{i}}{w_{j}}}}{1-\bigg({\huge\frac{w_{i}}{w_{j}}}\bigg)^{2Ne}} \qquad \qquad$} & Fixation probability\\
		\cline{2-2}
	\end{tabular}\\\
	
	\begin{tabular}{c r @{=} l}
		Where: & $Ne$ & Effective population size\\
	\end{tabular}\\
	
	\end{center}
	
	\begin{quote}
		This is the probability of fixation given a mutation occurs. \\
	\end{quote} 
	
	\begin{center}
	
	\begin{tabular}[c]{r |l|}
		\cline{2-2}
		{\Large $P_{j} = \mu_{ij} Ne \pi(i \rightarrow j) \qquad \qquad$} & Replacement probability\\
		\cline{2-2}
	\end{tabular}\\\
	
	\begin{tabular}{c r @{=} l}
		Where: & $\mu_{ij}$ & Mutation rate from codon i to codon j\\
	\end{tabular}\\

	\end{center}
	
	\begin{quote}
		This is the probability that a mutation occurs and it fixes. Basically, it is the probability
		of fixation given a mutation occurs, $\pi(i \rightarrow j)$, weighted by the rate at which mutations
		occur, $\mu_{ij} Ne$. \\
	\end{quote} 

%=================================================================================================================

\section{Other Info}
	
	\subsection{Compilers}
		The code can be recompiled from source and optimized for the hardware of the 
		machine it will be run on.\\	
		The source code has been successfully compiled with the following compilers.\\
		Mac:		gcc, g++\\
		Linux:		g++, gcc -lm\\	
		In addition, the development libraries of the GNU Scientific Libraries must be installed.\\
		
	\subsection{Build}
		Builds on Ubuntu 12.04 machine with GSL libraries using provided makefile.\\		
		Makefile provides several different sets of options. For use with multiple CPUs that share memory,
		choose a set of options that includes the flag -fopenmp when compiling\\
	
	\subsection{Bugs}
		In case of any bugs or trouble with the code, send an email to whewgley@utk.edu or mikeg@utk.edu\\
		
%=================================================================================================================	
\end{document}
